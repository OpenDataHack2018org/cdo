\section{Time axis}

A time axis describes the time for every timestep.
Two time types are available: absolute time and relative time.
{\CDO} tries to maintain the actual type of the time axis for all operators.
The operators for time range statistic (e.g.: monavg, ymonavg, ...)
create an absolute time axis.

\subsection{Absolute time}

An absolute time axis has the current time to each time step.
It can be used without knowledge of the calendar.
This is preferably used by climate models.
In netCDF files the relative time axis is represented by the 
unit of the time: {"{\tt day as \%Y\%m\%d.\%f}"}.

\subsection{Relative time}

A relative time is the time relative to a fixed reference time.
The current time results from the reference time and the elapsed interval.
The result depends on the calendar used.
{\CDO} supports the standard Gregorian, 360 days, 365 days and 366 days calendars.
The relative time axis is preferably used by weather forecast models.
In netCDF files the relative time axis is represented by the 
unit of the time: {"{\it time-units} {\tt since} {\it reference-time}"},
e.g "{\tt days since 1989-6-15 12:00}".

\subsection{Conversion of the time}

Some programs which work with netCDF data can only process relative time axes.
Therefore it may be necessary to convert from an absolute into a relative time axis.
This conversion can be done for each operator with the {\CDO} option '-r'.
%With the {\CDO} option '-r' can be made this conversion for each operator.
To convert a relative into an absolute time axis use the {\CDO} option '-a'.
