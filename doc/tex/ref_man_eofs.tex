

\newpage
\subsection{EOFS - Empirical Orthogonal Functions}
\index{eof}
\index{eoftime}
\index{eofspatial}
\label{EOFS}
\label{eof}
\label{eoftime}
\label{eofspatial}

\subsection*{Synopsis}

\hspace*{8mm}{$<\!operator\!>$}{\sl ,neof} \ {\tt ifile ofile1 ofile2}


\subsection*{Description}

\setlength{\miniwidth}{\textwidth}
\addtolength{\miniwidth}{-8mm}
\hspace*{8mm}\begin{minipage}{\miniwidth}
This module calculates empirical orthogonal functions of the data in {\tt ifile}.
If operator {\htmlref{eof}{eof}} is chosen, the EOFs are computed in either time or spatial
space, whichever is the fastest. If the user already knows, which computation
is faster, the module can be forced to perform a computation in time or gridspace
by using the operators {\htmlref{eoftime}{eoftime}} or {\htmlref{eofspatial}{eofspatial}}, respectively. This can enhance 
performance, especially for very long time series, where the number of time steps
is larger than the number of grid-points.
Data in {\tt ifile} are assumed to be anomalies. If they are not, the behavior 
of this module is {\bf not} well defined. 
After execution {\tt ofile1} will contain the eigen-values and {\tt ofile2} the
eigenvectors.
Missing value support is not fully supported. It is only checked for non-changing
masks of missing values in time. Although there still will be results, they are
not trustworthy, and a warning will occur. In the latter case we suggest to 
replace missing values by 0 in {\tt ifile} 
\end{minipage}
\addtolength{\miniwidth}{8mm}

\subsection*{Operators}

\setlength{\miniwidth}{\textwidth}
\addtolength{\miniwidth}{-8mm}
\hspace*{8mm}\begin{minipage}{\miniwidth}
\begin{defalist}{\bf eofspatial \ }
\item[{\bf eof}\ \ \hfill]
Calculate EOFs in spatial or time space \\
\item[{\bf eoftime}\ \ \hfill]
Calculate EOFs in time space \\
\item[{\bf eofspatial}\ \ \hfill]
Calculate EOFs in spatial space \\
\end{defalist}
\end{minipage}
\addtolength{\miniwidth}{8mm}

\subsection*{Parameter}

\setlength{\miniwidth}{\textwidth}
\addtolength{\miniwidth}{-8mm}
\hspace*{8mm}\begin{minipage}{\miniwidth}
\begin{defalist}{\sl neof \ }
\item[{\sl neof}\ \ \hfill]
\makebox[20mm][l]{\sf \small INTEGER}  Number of eigen functions
\end{defalist}
\end{minipage}
\addtolength{\miniwidth}{8mm}

\subsection*{Example}

\setlength{\miniwidth}{\textwidth}
\addtolength{\miniwidth}{-8mm}
\hspace*{8mm}\begin{minipage}{\miniwidth}
To calculate the first 40 EOFs of a data-set containing anomalies use:
\begin{lstlisting}[backgroundcolor=\color{zebg}, basicstyle=\small]
   cdo eof,40 ifile ofile1 ofile2
\end{lstlisting}

If the dataset does not containt anomalies, process them first,
and use:
\begin{lstlisting}[backgroundcolor=\color{zebg}, basicstyle=\small]
   cdo sub ifile1 -timmean ifile1 anom_file	
   cdo eof,40 anom_file ofile1 ofile2	
\end{lstlisting}
\end{minipage}
\addtolength{\miniwidth}{8mm}


\newpage
\subsection{EOFCOEFF - Principal coefficients of EOFs}
\index{eofcoeff}
\label{EOFCOEFF}
\label{eofcoeff}

\subsection*{Synopsis}

\hspace*{8mm}{\bf eofcoeff}{\sl } \ {\tt ifile1 ifile2 obase}


\subsection*{Description}

\setlength{\miniwidth}{\textwidth}
\addtolength{\miniwidth}{-8mm}
\hspace*{8mm}\begin{minipage}{\miniwidth}
This module calculates the time series of the principal coefficients for given EOF
(empirical orthogonal functions) and data. Time steps in {\tt ifile1} are 
assumed to be the EOFs, Time steps in {\tt ifile2} are assumed to be the
time series. 
There will be a separate file containing a time series of principal coefficients
with time information from {\tt ifile2} for each EOF in {\tt ifile1}. Files
will be numbered as {\tt $<$obase$>$\_$<$neof$>$.$<$suffix$>$} where neof is the number of the
EOF (time step) in {\tt ifile1} and suffix is a file-suffix derived from the 
current file format. 
\end{minipage}
\addtolength{\miniwidth}{8mm}
\setlength{\miniwidth}{\textwidth}
\addtolength{\miniwidth}{-8mm}
\hspace*{8mm}\begin{minipage}{\miniwidth}
\end{minipage}
\addtolength{\miniwidth}{8mm}

\subsection*{Example}

\setlength{\miniwidth}{\textwidth}
\addtolength{\miniwidth}{-8mm}
\hspace*{8mm}\begin{minipage}{\miniwidth}
To calculate principal coefficients of the first 40 EOFs of ifile, 
and write the to ofile, use:
\begin{lstlisting}[backgroundcolor=\color{zebg}, basicstyle=\small]
   cdo eof,40 anom_file evec_file eof_file
   cdo eofcoeff eof_file anom_file obase
\end{lstlisting}
The principal coefficients of the first EOF will be in the file
{\tt obase\_000000.nc}

If the dataset {\tt ifile} does not containt anomalies, process them first,
and use:
\begin{lstlisting}[backgroundcolor=\color{zebg}, basicstyle=\small]
   cdo sub ifile1 -timmean ifile1 anom_file	
   cdo eof,40 anom_file evec_file eof_file
   cdo eofcoeff eof_file anom_file obase 	
\end{lstlisting}
\end{minipage}
\addtolength{\miniwidth}{8mm}
