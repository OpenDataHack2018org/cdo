%\chapter{\label{refman}ECA indices of extremes}
\chapter{\label{refman}Climate indices reference manual}

This section gives a description of all {\CDO} operators to compute the climate indices of daily temperature and precipitation extreme.
Related operators are grouped to modules.
For easier description all single input files are named {\tt ifile} or {\tt ifile1}, {\tt ifile2}, etc.,
and an arbitrary number of input files are named {\tt ifiles}.
All output files are named {\tt ofile} or {\tt ofile1}, {\tt ofile2}, etc.
Further the following notion is introduced:
\begin{defalist}{{\em o(t,x)}}
\item[\(i(t)\)\hfill]
Timestep \(t\) of {\tt ifile}
\item[\(i(t,x)\)\hfill]
Element number \(x\) of the field at timestep \(t\) of {\tt ifile}
\item[\(o(t)\)\hfill]
Timestep \(t\) of {\tt ofile}
\item[\(o(t,x)\)\hfill]
Element number \(x\) of the field at timestep \(t\) of {\tt ofile}
\end{defalist}

%A field is a horizontal slice of a variable on a spezific level.
%The number of elements is the size of the grid.
%It has at least one datapoint. The rank of a field is 1 or 2, this depends
%on the type of the grid.

%A variable is a collection of all fields on different vertical levels
%of the same spezies. Each variable has at least on level. Variable
%with only one level are called 2D variables and all other are 3D variables.

\hspace{3mm}

%\newpage
%\section{Climate indices of daily temperature and precipitation extremes}
%This section contains modules to compute the climate indices
of daily temperature and precipitation extremes.


\input{ref_list_climat}
\input{ref_man_climat}
