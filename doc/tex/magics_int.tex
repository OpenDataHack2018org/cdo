\chapter{Introduction}

The Climate Data Operators ({\CDO}) software is a collection of operators
for standard processing of climate and forecast model data.

This document describes additional {\CDO} operators to ....

Magics++ is the latest generation of the ECMWF's Meteorological plotting software MAGICS.
Magics++ supports the plotting of contours, wind fields, observations, satellite images, symbols, text,
axis and graphs (including box plots). Data fields to be plotted may be presented in various formats, for
instance GRIB 1 and 2 code data, gaussian grid, regularly spaced grid and fitted data,BUFR and
NetCDF format or retrieved from an ODB database. The produced meteorological plots can be saved in
various formats, such as PostScript, EPS, PDF, GIF, PNG and SVG.[1]

In order to rapidly generate high quality pictures from the data obtained from the existing CDO
operators, the CDO has been interfaced with Magics++ library. As a first step,some new CDO plotting
operators are created to cater to the most essential/ frequently used plotting features viz., graph,
contour, vector. These operators rely on the Magics++ and generate output files in the various formats
supported by Magics++.

Magics++ provides a vast number of parameters to control the attributes of various plotting
features. Keeping in view, the usability of CDO users, currently only a few of these parameters are
supported and accessible to the CDO users as command line arguments for the respective operators.
The users are requested to refer to the Magics++ manual[1] for detailed description of the various
parameters available for the various features. An option is provided to CDO users to input some of
these parameters in an XML file which can be passed as an argument for the CDO plotting operators.
The description of the new plotting operators and the various arguments that can be passed for these
operators is provided in the subsequent sections.
