\chapter{Introduction}

The Climate Data Operators ({\CDO}) software is a collection of operators
for standard processing of climate and forecast model data.

This document describes additional {\CDO} operators to be used for generating plots.

Magics is the latest generation of the ECMWF's Meteorological plotting software MAGICS.
Magics supports the plotting of contours, wind fields, observations, satellite images, symbols, text,
axis and graphs (including box plots). Data fields to be plotted may be presented in various formats, for
instance GRIB 1 and 2 code data, gaussian grid, regularly spaced grid and fitted data, BUFR and
NetCDF format or retrieved from an ODB database. The produced meteorological plots can be saved in
various formats, such as PostScript, EPS, PDF, GIF, PNG and SVG. \cite{Magics}

In order to rapidly generate high quality pictures from the data obtained from the existing {\CDO}
operators, the {\CDO} has been interfaced with the Magics library. As a first step, some {\CDO} plotting
operators are created to cater to the most essential/ frequently used plotting features viz., graph,
contour, vector. These operators rely on the Magics and generate output files in the various formats
supported by Magics. These operators can be used as terminal operators and chained with the existing 
operators.

Magics provides a vast number of parameters to control the attributes of various plotting
features. Keeping in view, the usability of {\CDO} users, currently only a few of these parameters are
supported and accessible to the {\CDO} users as command line arguments for the respective operators.
The users are requested to refer to the Magics manual \cite{Magics} for detailed description of the various
parameters available for the various features. The description of the plotting operators and the
various arguments that can be passed to these operators is provided in the subsequent sections.

\chapter{Building CDO with Magics}

This section describes how to build and install {\CDO} with Magics
support on a UNIX system.

\section{Magics}
 
Magics needs to be installed before building {\CDO}.
The Fortran and netCDF support can be disabled when building Magics from source.

\section{Compilation}

First go to the {\CDO}  \href{https://code.zmaw.de/projects/cdo}{\tt download} page
({\tt https://code.zmaw.de/projects/cdo}) to get the latest distribution,
if you do not have it yet.
Compilation is done by performing the following steps:

\begin{enumerate}
\item Unpack the archive, if you haven't done that yet:
   
\begin{verbatim}
    gunzip cdo-$VERSION.tar.gz    # uncompress the archive
    tar xf cdo-$VERSION.tar       # unpack it
    cd cdo-$VERSION
\end{verbatim}

\item Configure {\CDO} with Magics support:
 
\begin{verbatim}
./configure --with-magics=<Magics root directory>
\end{verbatim}

For an overview of other configuration options use

\begin{verbatim}
    ./configure --help
\end{verbatim}

\item Compile the program by running make:

\begin{verbatim}
    make
\end{verbatim}

\end{enumerate}

The program should compile without problems and the binary ({\tt cdo}) 
should be available in the {\tt src} directory of the distribution.


\section{Installation}

After the compilation of the source code do a {\tt make install},
possibly as root if the destination permissions require that.

\begin{verbatim}
    make install
\end{verbatim} 

The binary is installed into the directory {\tt $<$prefix$>$/bin}.
{\tt $<$prefix$>$} defaults to {\tt /usr/local} but can be changed with 
the {\tt --prefix} option of the configure script. 
