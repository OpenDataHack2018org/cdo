\section{Percentile}

There is no standard definition of percentile.
All definitions yield to similar results when the number of values is very large.
The following percentile methods are available in {\CDO}:

\vspace{2mm}
\hspace{2cm}
\begin{tabular}[c]{|>{\columncolor{pcolor1}}l|l|}
\hline
\rowcolor{pcolor1}
\cellcolor{pcolor2}
Percentile            & \\
\rowcolor{pcolor1}
\cellcolor{pcolor2}
method               & \multirow{-2}{*}{Description} \\
\hline
 nrank                  & Nearest Rank method, the default method used in {\CDO} \\
\hline
 nist                     & The primary method recommended by NIST \\
\hline
 numpy                & numpy.percentile with the option interpolation set to 'linear' \\
\hline
 numpy\_lower    & numpy.percentile with the option interpolation set to 'lower' \\
\hline
 numpy\_higher   & numpy.percentile with the option interpolation set to 'higher' \\
\hline
 numpy\_nearest & numpy.percentile with the option interpolation set to 'nearest' \\
\hline
\end{tabular}

\vspace{3mm}

The percentile method can be selected with the {\CDO} option \texttt{-\,-percentile}.
The Nearest Rank method is the default percentile method in {\CDO}.

The different percentile methods can lead to different results,
especially for small number of data values.
Consider the ordered list \{15, 20, 35, 40, 50, 55\}, which contains six
data values.
Here is the result for  the 30th, 40th, 50th, 75th and 100th percentiles of
this list using the different percentile methods:

\vspace{2mm}
\hspace{2cm}
\begin{tabular}[c]{|>{\columncolor{pcolor1}}c|c|c|c|c|c|c|}
\hline
\rowcolor{pcolor1}
\cellcolor{pcolor2}
Percentile &      &  & & numpy & numpy & numpy \\
\rowcolor{pcolor1}
\cellcolor{pcolor2}
 P              &  \multirow{-2}{*}{nrank}  &  \multirow{-2}{*}{nist}  &   \multirow{-2}{*}{numpy}  & lower   & higher & nearest \\
\hline
30th              &   20   &   21.5    &    27.5      &  20      &   35  &  35  \\
\hline
40th              &   35   &   32       &    35        &  35      &   35   & 35 \\
\hline
50th              &   35    &  37.5     &   37.5      &  35      &  40    & 40  \\
\hline
75th              &   50    &  51.25     &   47.5    &  40      &  50    & 50 \\
\hline 
100th            &   55    &  55          &   55        &  55      &  55     &  55 \\
\hline
\end{tabular}

\vspace{3mm}

\subsection{Percentile over timesteps}

The amount of data for time series can be very large.
All data values need to held in memory to calculate the percentile.
The percentile over timesteps uses a histogram algorithm, to limit the
amount of required memory. The default number of histogram bins is 101.
That means the histogram algorithm is used, when the dataset has more than 101 time steps.
The default can be overridden by setting the environment variable \texttt{CDO\_PCTL\_NBINS} to a different value.
The histogram algorithm is implemented only for the Nearest Rank method.
