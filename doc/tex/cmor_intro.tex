\chapter{Introduction}

The Climate Data Operators ({\CDO}) software is a collection of operators
for standard processing of climate and forecast model data. Here, the single
operator \textbf{cdo cmor} is documented which demands a separate explanation 
because of its many options to rearrange model output and add Metadata.
It represents an interface to the Climate Model Output Rewriter library \href{https://pcmdi.github.io/cmor-site/}{CMOR} developed at PCMDI. This library comprises a set of functions which can be used to produce NetCDF files that fulfill the requirements of many of the climate community's standard model experiments. The output resulting from CMOR is "self-describing" and
facilitates analysis of results across models. 

CMOR functions feature a specific processing and require much input. 
The objective of \textbf{cdo cmor} is to provide an easy interface to CMOR in order
to automate and simplify its usage.
\textbf{cdo cmor} can guarantee the right CMOR configuration by calling the
CMOR functions internally in the right order in which CMOR labels and input attributes
are inserted correctly. A reduction of input can be achieved by exploiting 
the information gained by the climate data interface CDI. The final \textbf{cdo cmor} has a minimum requirement of one argument and an input file to apply CMOR accurately

Different versions of the CMOR library have been published over the years. 
The operator is based on CMOR version 2 which is appropriate to build CMIP5 compliant netcdf model output. The data standard of CMIP5 is fixed so that the project's database could be used to validate and improve the operator's functionality. Besides, there will be no update to another CMOR 2 version which guarantees a stable source package.

This documentation is about the operator which uses CMOR version 2. In next {\CDO} releases, the operator will be upgraded by the support for more recent CMOR versions in order to achieve the long term goal of facilitating the preparation of project compliant output for all CMIP phases and other projects. 

The  {\CDO} operator \textbf{cmor} was developed at the DKRZ and MPI for
Meteorology and will be enhanced at the DKRZ for the CMIP6 project. 