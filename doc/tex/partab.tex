\section{Parameter table}

A parameter table is an ASCII formated file to convert code numbers to variable names.
Each variable has one line with the code number, the name and the description
with optional units in a blank separated list.
It can be used only for GRIB, SERVICE, EXTRA and IEG formated files.
The {\CDO} option '-t $<$partab$>$' sets the default parameter table for all input files.
Use the operator 'setpartab' to set the parameter table for a specific file.

\vspace{2mm}

\begin{minipage}[t]{\textwidth}
Example of a {\CDO} parameter table:
\begin{lstlisting}[frame=single, backgroundcolor=\color{zebg}, basicstyle=\footnotesize]
    134  aps      surface pressure [Pa]
    141  sn       snow depth [m]
    147  ahfl     latent heat flux [W/m**2]
    172  slm      land sea mask
    175  albedo   surface albedo
    211  siced    ice depth [m]
\end{lstlisting}
\end{minipage}

%%% Local Variables: 
%%% mode: latex
%%% TeX-master: "missval"
%%% End: 
