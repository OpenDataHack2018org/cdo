\section{Building from sources}

This section describes how to build {\CDO} from the sources on a UNIX system.
{\CDO} uses the GNU configure and build system for compilation.
The only requirement is a working ISO C++11 and ANSI C99 compiler.

%First go to the \href{http://www.mpimet.mpg.de/cdo/download.html}{\texttt{download}} page
%(\texttt{http://www.mpimet.mpg.de/\\,cdo/download.html}) to get the latest distribution,
%First go to the \href{http://www.mpimet.mpg.de/cdo}{\texttt{download}} page
%(\texttt{http://www.mpimet.mpg.de/cdo}) to get the latest distribution,
First go to the \href{https://code.zmaw.de/projects/cdo}{\texttt{download}} page
(\texttt{https://code.zmaw.de/projects/cdo}) to get the latest distribution,
if you do not have it yet.

To take full advantage of {\CDO} features the following additional libraries should be installed:

\begin{itemize}
\item Unidata \href{https://www.unidata.ucar.edu/software/netcdf}{NetCDF} library
      (\texttt{https://www.unidata.ucar.edu/software/netcdf})
      version 3 or higher. \\
      This library is needed to process NetCDF \cite{NetCDF} files with {\CDO}. 
\item ECMWF \href{https://software.ecmwf.int/wiki/display/ECC/ecCodes+Home}{ecCodes} library
      (\texttt{https://software.ecmwf.int/wiki/display/ECC/ecCodes+Home})
      version 2.3.0 or higher.
      This library is needed to process GRIB2 files with {\CDO}. 
\item HDF5 \href{http://www.hdfgroup.org/doc_resource/SZIP}{szip} library
      (\texttt{http://www.hdfgroup.org/doc\_resource/SZIP})
      version 2.1 or higher. \\
      This library is needed to process szip compressed GRIB \cite{GRIB} files with {\CDO}. 
\item \href{http://www.hdfgroup.org/HDF5}{HDF5} library
      (\texttt{http://www.hdfgroup.org/HDF5})
      version 1.6 or higher. \\
      This library is needed to import  CM-SAF \cite{CM-SAF} HDF5 files with the {\CDO}
      operator \textbf{import\_cmsaf}. 
\item \href{http://trac.osgeo.org/proj}{PROJ.4} library
      (\texttt{http://trac.osgeo.org/proj})
      version 4.6 or higher. \\
      This library is needed to convert Sinusoidal and Lambert Azimuthal Equal Area coordinates
      to geographic coordinates, for e.g. remapping. 
\item \href{https://software.ecmwf.int/wiki/display/MAGP/Magics}{Magics} library
      (\texttt{https://software.ecmwf.int/wiki/display/MAGP/Magics})
      version 2.18 or higher. \\
      This library is needed to create contour, vector and graph plots with {\CDO}.
\end{itemize}

{\CDO} is a multi-threaded application. Therefor all the above libraries should be compiled thread safe. 
Using non-threadsafe libraries could cause unexpected errors!

%This section is divided into the following sections:
%<ul>
%<li>\ref install_src_unix  "Compiling from source on Unix"
%<li>\ref install_bin_unix  "Installing the binaries on Unix"
%<li>\ref build_tools       "Tools used to develop CDO"
%</ul>


\subsection{Compilation}

Compilation is done by performing the following steps:

\begin{enumerate}
\item Unpack the archive, if you haven't done that yet:
   
\begin{verbatim}
    gunzip cdo-$VERSION.tar.gz    # uncompress the archive
    tar xf cdo-$VERSION.tar       # unpack it
    cd cdo-$VERSION
\end{verbatim}
%$
\item Run the configure script:
 
\begin{verbatim}
    ./configure
\end{verbatim}

\begin{itemize}
\item Optionaly with NetCDF \cite{NetCDF} support:
 
\begin{verbatim}
./configure --with-netcdf=<NetCDF root directory>
\end{verbatim}

\item and with ecCodes:
 
\begin{verbatim}
./configure --with-eccodes=<ecCodes root directory>
\end{verbatim}

%%You have to specify also the location of the SZLIB if HDF5 was build with SZLIB support.\\
%%
%%\item To enable szip \cite{szip} support add:
%% 
%%\begin{verbatim}
%%    --with-szlib=<SZLIB root directory>
%%\end{verbatim}

\end{itemize}

For an overview of other configuration options use

\begin{verbatim}
    ./configure --help
\end{verbatim}

\item Compile the program by running make:

\begin{verbatim}
    make
\end{verbatim}

\end{enumerate}

The program should compile without problems and the binary (\texttt{cdo}) 
should be available in the \texttt{src} directory of the distribution.


\subsection{Installation}

After the compilation of the source code do a \texttt{make install},
possibly as root if the destination permissions require that.

\begin{verbatim}
    make install
\end{verbatim} 

The binary is installed into the directory \texttt{$<$prefix$>$/bin}.
\texttt{$<$prefix$>$} defaults to \texttt{/usr/local} but can be changed with 
the \texttt{--prefix} option of the configure script. 

Alternatively, you can also copy the binary from the \texttt{src} directory
manually to some \texttt{bin} directory in your search path.
