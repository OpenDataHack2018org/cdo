\chapter{\label{refman}Reference manual}
%\chapter{\label{refman}Reference manual for all operators}

This section gives a description of all operators. Related operators are grouped to modules.
For easier description all single input files are named \texttt{ifile} or \texttt{ifile1}, \texttt{ifile2}, etc.,
and an arbitrary number of input files are named \texttt{ifiles}.
All output files are named \texttt{ofile} or \texttt{ofile1}, \texttt{ofile2}, etc.
Further the following notion is introduced:
\begin{defalist}{{\em o(t,x)}}
\item[\(i(t)\)\hfill]
Timestep \(t\) of \texttt{ifile}
\item[\(i(t,x)\)\hfill]
Element number \(x\) of the field at timestep \(t\) of \texttt{ifile}
\item[\(o(t)\)\hfill]
Timestep \(t\) of \texttt{ofile}
\item[\(o(t,x)\)\hfill]
Element number \(x\) of the field at timestep \(t\) of \texttt{ofile}
\end{defalist}

%A field is a horizontal slice of a variable on a spezific level.
%The number of elements is the size of the grid.
%It has at least one datapoint. The rank of a field is 1 or 2, this depends
%on the type of the grid.

%A variable is a collection of all fields on different vertical levels
%of the same spezies. Each variable has at least on level. Variable
%with only one level are called 2D variables and all other are 3D variables.

\newpage
\section{Information}
This section contains modules to print information about datasets.
All operators print there results to standard output.

\input{ref_list_inform}
\input{ref_man_inform}

\newpage
\section{File operations}
This section contains modules to perform operations on files.

\input{ref_list_file_o}
\input{ref_man_file_o}

\newpage
\section{Selection}
This section contains modules to select time steps, fields or part
of a field from a dataset.

\input{ref_list_select}
\input{ref_man_select}

\newpage
\section{Conditional selection}
This section contains modules to conditional select field elements.
The fields in the first input file are handled as a mask. 
A value not equal to zero is treated as "true", zero is treated as "false".

\input{ref_list_condit}
\input{ref_man_condit}

\newpage
\section{Comparison}
This section contains modules to compare datasets.
The resulting field is a mask containing 1 if the comparison is true
and 0 if not.

\input{ref_list_compar}
\input{ref_man_compar}

\newpage
\section{Modification}
This section contains modules to modify the metadata, fields or 
part of a field in a dataset.

\input{ref_list_modifi}
\input{ref_man_modifi}

\newpage
\section{Arithmetic}
This section contains modules to arithmetically process datasets.

\input{ref_list_arithm}
\input{ref_man_arithm}

\newpage
\section{Statistical values}
This section contains modules to compute statistical values of datasets.
In this program there is the different notion of "mean" and "average"
to distinguish two different kinds of treatment of missing values.
While computing the mean, only the not missing values are considered
to belong to the sample with the side effect of a probably reduced sample
size. Computing the average is just adding the sample members and divide
the result by the sample size. For example, the mean of 1, 2, miss and 3
is (1+2+3)/3 = 2, whereas the average is (1+2+miss+3)/4 = miss/4 = miss.
If there are no missing values in the sample, the average and the mean are
identical.
In this section the abbreviations as in the following table are used:

\vspace{3mm}

\fbox{\parbox{15cm}{
\begin{eqnarray*}
\begin{array}{l}
\makebox[3cm][l]{{\bf sum}} \\
\end{array}
 &  &
\sum_{i=1}^{n} x_i 
\\
\begin{array}{l}
\makebox[3cm][l]{{\bf mean} resp. {\bf avg}} \\
\end{array}
 &  &
 n^{-1} \sum_{i=1}^{n} x_i 
\\
\begin{array}{l}
\makebox[3cm][l]{{\bf mean} resp. {\bf avg}} \\
\mbox{weighted by} \\
\{w_i, i=1, ..., n\}  \\
\end{array}
 & &
  \left ( \sum_{j=1}^{n} w_j \right )^{-1} \sum\limits_{i=1}^{n} w_i \, x_i \\
\\
\begin{array}{l}
\makebox[3cm][l]{Variance} \\
\makebox[3cm][l]{{\bf var}} \\
\end{array}
 &  &
 n^{-1} \sum_{i=1}^{n} (x_i - \overline{x})^2
\\
\begin{array}{l}
\makebox[3cm][l]{{\bf var} weighted by} \\
\{w_i, i=1, ..., n\}  \\
\end{array}
 & &
  \left ( \sum_{j=1}^{n} w_j \right )^{-1} \sum\limits_{i=1}^{n} w_i \, 
  \left ( x_i - \left ( \sum_{j=1}^{n} w_j \right )^{-1} \sum\limits_{j=1}^{n} w_j \, x_j \right)^2 \\
\\
\begin{array}{l}
\makebox[3cm][l]{Standard deviation} \\
\makebox[3cm][l]{{\bf std}} \\
\end{array}
 &  &
\sqrt{ n^{-1} \sum_{i=1}^{n} (x_i - \overline{x})^2 }
\\
\begin{array}{l}
\makebox[3cm][l]{{\bf std} weighted by} \\
\{w_i, i=1, ..., n\}  \\
\end{array}
 & &
\sqrt{
  \left ( \sum_{j=1}^{n} w_j \right )^{-1} \sum\limits_{i=1}^{n} w_i \, 
  \left ( x_i - \left ( \sum_{j=1}^{n} w_j \right )^{-1} \sum\limits_{j=1}^{n} w_j \, x_j \right)^2 } \\
\end{eqnarray*}
}}
\\


\input{ref_list_statis}
\input{ref_man_statis}

\newpage
\section{Correlation}
This sections contains modules for correlation in space and time.

\input{ref_list_correl}
\input{ref_man_correl}

\newpage
\section{Regression}
This sections contains modules for linear regression of time series.

\input{ref_list_regres}
\input{ref_man_regres}

\newpage
\section{Interpolation}
This section contains modules to interpolate datasets.
There are several operators to interpolate horizontal fields
to a new grid. Some of those operators can handle only
2D fields on a regular rectangular grid.
Vertical interpolation of 3D variables is possible from
hybrid model levels to height or pressure levels.
Interpolation in time is possible between time steps and years.

\input{ref_list_interp}
\input{ref_man_interp}

\newpage
\section{Transformation}
This section contains modules to perform spectral transformations.

\input{ref_list_transf}
\input{ref_man_transf}

\newpage
\section{Formatted I/O}
This section contains modules to read and write ASCII data.

\input{ref_list_format}
\input{ref_man_format}

\newpage
\section{Miscellaneous}
This section contains miscellaneous modules
which do not fit to the other sections before.

\input{ref_list_miscel}
\input{ref_man_miscel}

\newpage
\section{Climate indices}
This section contains modules to compute the climate indices
of daily temperature and precipitation extremes.


\input{ref_list_climat}
\input{ref_man_climat}


