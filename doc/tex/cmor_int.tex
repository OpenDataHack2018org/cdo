\chapter{Introduction}

The Climate Data Operators ({\CDO}) software is a collection of operators
for standard processing of climate and forecast model data.

This document describes the additional {\CDO} operator \textbf{cmor}. This
operator is an interface to the CMOR library version 3 from PCMDI.
The CMOR support is available with {\CDO} release 1.8.0 or later.

The \cite{CMOR} (Climate Model Output Rewriter) library comprises a set of
functions, that can be used to produce CF-compliant NetCDF files that 
fulfill the requirements of many of the climate community's standard
model experiments. These experiments are collectively referred to as
MIP's and include, for example, AMIP, CMIP, CFMIP, PMIP, APE, and IPCC 
scenario runs. The output resulting from CMOR is "self-describing" and
facilitates analysis of results across models.

Much of the metadata written to the output files is defined in
MIP-specific tables, typically made available from each MIP's web
site. CMOR relies on these tables to provide much of the metadata 
that is needed in the MIP context, thereby reducing the programming 
effort required of the individual MIP contributors.

The  {\CDO} operator \textbf{cmor} was developed at the DKRZ and MPI for
Meteorology to provide an easy interface to the \cite{CMOR} library for a
standardized preparation of CMIP6 data.

\chapter{Building CDO with CMOR}

This section describes how to build and install {\CDO} with CMOR
support on a UNIX system.

\section{CMOR library}
 
CMOR version 3 needs to be installed before building {\CDO}.
The CMOR library depends on the following external libraries:
netCDF4, HDF5, UDUNITS2, zlib and uuid.
Make sure that you use exactly the same libraries for the {\CDO}
installation, otherwise the operator \textbf{cmor} will possibly not working correctly.

\section{Compilation}

First go to the {\CDO}  \href{https://code.mpimet.mpg.de/projects/cdo}{\texttt{download}} page
(\texttt{https://code.mpimet.mpg.de/projects/cdo}) to get the latest distribution,
if you do not have it yet.
Compilation is done by performing the following steps:

\begin{enumerate}
\item Unpack the archive, if you haven't done that yet:
   
\begin{verbatim}
    gunzip cdo-$VERSION.tar.gz    # uncompress the archive
    tar xf cdo-$VERSION.tar       # unpack it
    cd cdo-$VERSION
\end{verbatim}

\item Configure {\CDO} with CMOR support:

The configuration depends on the location of the external libraries. Here is one example:

\begin{verbatim}
./configure --with-cmor=<CMOR root directory> --with-netcdf=<NetCDFroot directory> \
            --with-uuid --with-udunits2 LIBS=-lossp-uuid                           \
            CPPFLAGS="-I<CMOR root dir>/include/cdTime -I<CMOR root dir>/include/json-c"
\end{verbatim}

For an overview of other configuration options use

\begin{verbatim}
    ./configure --help
\end{verbatim}

\item Compile the program by running make:

\begin{verbatim}
    make
\end{verbatim}

\end{enumerate}

The program should compile without problems and the binary (\texttt{cdo}) 
should be available in the \texttt{src} directory of the distribution.


\section{Installation}

After the compilation of the source code do a \texttt{make install},
possibly as root if the destination permissions require that.

\begin{verbatim}
    make install
\end{verbatim} 

The binary is installed into the directory \texttt{$<$prefix$>$/bin}.
\texttt{$<$prefix$>$} defaults to \texttt{/usr/local} but can be changed with 
the \texttt{--prefix} option of the configure script. 
